% EXEMPLE DE CONTRIBUTION AU GRETSI

% Classe `gretsi`
% Définir la langue utilisée (`francais` ou `english`)
\documentclass[francais]{gretsi}

% Encodage du fichier source
\usepackage[utf8]{inputenc}

% Encodage des caractères dans le PDF
\usepackage[T1]{fontenc}

% Packages mathématiques
\usepackage{amsmath, amssymb, amsfonts}

\begin{document}

% ===================================================================================== %

% Titre
\titre{Instructions aux auteurs du Gretsi \\ Format \LaTeXe}

% Auteurs
\auteurs{
  % Syntaxe : \auteur{<prénom>}{<nom>}{<adresse électronique>}{<indice d'affiliation>}
  \auteur{Michel}{Dupont}{mdupont@uni.fr}{1}
  \auteur{Danielle}{Durand}{dd@uni.com}{2}
  \auteur{Patrick}{Martin}{pm@uni.fr}{2}
}

% Affiliations
\affils{
  % Syntaxe : \affil{<indice d'affiliation>}{<nom du labo et adresse>}
  \affil{1}{Laboratoire Traitement des Signaux,
        1 rue de la parole, BP 00000,
        99000 Nouvelleville Cedex 00, France
  }
  \affil{2}{Laboratoire Traitement des Images,
        1 rue de la vision, BP 99999,
        00000 Autreville, France
  }
}

% Si tous les auteurs ont la même adresse : ne pas mettre de numéro d'affiliation. Exemple :
% \auteurs{
  % \auteur{Michel}{Dupont}{mdupont@uni.fr}{}
  % \auteur{Danielle}{Durand}{dd@uni.com}{}
% }
% \affils{
%   \affil{}{Laboratoire Traitement des Signaux,
%         1 rue de la parole, BP 00000,
%         99000 Nouvelleville Cedex 00, France
%   }
% }

% Résumé en français
\resume{Les auteurs publiant au GRETSI et utilisant le traitement de texte \LaTeXe\
trouveront ci-dessous quelques indications destinées à leur faciliter la tâche.
Le fichier \texttt{gretsi.tex} qui contient le présent document
respecte les contraintes fixées.}

% Résumé en anglais
\abstract{GRETSI authors who are \LaTeXe\ users
will find above some informations to help them.
The file \texttt{gretsi.tex} which contains this document obeys the rules.}

\maketitle

% ===================================================================================== %

\section{Format du document}

\subsection{La classe \texttt{gretsi}}

Votre article ne doit pas dépasser 4 pages, tableaux et figures inclus.
Il est constitué de deux colonnes de 86~mm, espacées de 10~mm.
La classe \texttt{gretsi.cls} au format \LaTeXe\ que nous vous recommandons d'utiliser
vous permettra de réaliser automatiquement la mise en page, à l'aide de la commande~:
\begin{verbatim}
  \documentclass[francais]{gretsi}
\end{verbatim}
L'option \texttt{francais} gère tous les aspects liés à la rédaction et la typographie française.
l'article peut être écrit en anglais pour les auteurs non francophones~;
dans ce cas, utilisez l'option \texttt{english}.

Le paquet \texttt{gretsi} charge automatiquement les paquets suivants~:
\begin{center}
  \begin{tabular}{l@{\hspace*{15mm}}l}
    \texttt{algorithm2e}, &
    \texttt{babel}, \\
    \texttt{booktabs}, &
    \texttt{geometry}, \\
    \texttt{graphicx}, &
    \texttt{hyperref}, \\
    \texttt{microtype}, &
    \texttt{times}. \\
  \end{tabular}
\end{center}

D'autres paquets sont à charger dans le code source de votre article,
en modifiant si nécessaire les options~:
\begin{verbatim}
  \usepackage[utf8]{inputenc}
  \usepackage[T1]{fontenc}
  \usepackage{amsmath, amssymb, amsfonts}
\end{verbatim}

\subsection{En-tête de l'article}

En début de document, après \verb!\begin{document}!, vous devrez définir les informations suivantes~:
\begin{itemize}

  \item le titre de l'article~:\\
  \verb!\titre{Titre de l'article}!

  \item le prénom et le nom de chaque auteur, suivi d'un numéro renvoyant à son adresse~:\\
  \verb!\auteurs{!\\
  \verb!   \auteur{Pierre}{Dupont}!\\
  \verb!          {p.dupont@gretsi.fr}{1},!\\
  \verb!   \auteur{John}{Smith}!\\
  \verb!          {john.smith@gretsi.fr}{2}}!

  \item l'adresse de chaque auteur~:\\
  \verb!\affils{\affil{1}{Labo, rue, ville}! \\
  \verb!        \affil{2}{Labo, rue, ville}!

  \item les résumés en français et en anglais~:\\
  \verb!\resume{Résumé en français}! \\
  \verb!\abstract{Abstract in English}!

  \item enfin, indiquez à \LaTeXe de créer le titre~: \\
  \verb!\maketitle! \\

\end{itemize}

\subsection{Titre et sous-titres}

La hiérarchie du document est obtenue avec les commandes \verb!\section! et \verb!\subsection!.

% ===================================================================================== %

\section{Langue et typographie}

La typographie de votre article doit respecter les règles habituelles de la langue de rédaction.

En français, la ponctuation haute (\verb&;& \verb&:& \verb&?& \verb&!&) doit être précédée d'une espace insécable
(codée par un tilde).
Les guillemets à la française sont entrées avec
\verb!\og{}texte\fg{}! ou
\verb!<< texte >>!.
Enfin, les majuscules doivent être accentuées.

% ===================================================================================== %

\section{Tableaux}

Les tableaux doivent être précédés par leur légende (\verb!\caption!),
un exemple est donné table~\ref{puissancededeux}.

\begin{table}[htb]
    \caption{\label{puissancededeux}table des puissances}
    \begin{center}
    \begin{tabular}{l*{8}{c}}
        \toprule
        $n$   &   1 &   2 &   3 &   4 &   5 &    6 &  7 &  8 \\
        \midrule
        $n^2$ &   1 &   4 &   9 &  16 &  25 &  36 &  49 &  64 \\
        $n^3$ &   1 &   8 &  27 &  64 & 125 & 216 & 343 & 512 \\
        \bottomrule
    \end{tabular}
    \end{center}
\end{table}

% ===================================================================================== %

\section{Figures}

À l'inverse, les figures sont suivies par leur titre, comme c'est le cas de la figure~\ref{logo}.

\begin{figure}[htb]
  \begin{center}
    \includegraphics[width=.5\columnwidth]{logo}
  \end{center}
  \caption{Logo du Gretsi (ici, de taille égale à la moitié de la taille de la colonne).}
  \label{logo}
\end{figure}

% ===================================================================================== %

\section{Formules mathématiques}

Les formules mathématiques sont numérotées, comme c'est le cas de la formule \ref{formule}~:
\begin{equation}
  \label{formule}
  F(x) = \int_{-\infty}^x f(u)\,du.
\end{equation}

% ===================================================================================== %

\section{Algorithmes}

Les algorithmes sont insérés avec l'environnement \href{https://distrib-coffee.ipsl.jussieu.fr/pub/mirrors/ctan/macros/latex/contrib/algorithm2e/doc/algorithm2e.pdf}{\texttt{algorithm2e}}.
Les lignes peuvent être numérotées avec la commande \verb!\LinesNumbered!.

\LinesNumbered
\begin{algorithm}
  $i = 0$ \\
  \Pour{$x=0...10$}{
    $i = i + 1$ \\
  }
  \caption{Exemple d'algorithme}
\end{algorithm}

% ===================================================================================== %

\section{Bibliographie}

La bibliographie est insérée avec Bib\TeX.
Dans l'article, utilisez la commande \verb!\bibliography!.
Le style de bibliographie n'est pas à préciser car il est automatiquement défini dans la classe~:
il s'agit de \texttt{plain-fr} (en français) ou \texttt{plain} (en anglais).

% ===================================================================================== %

\bibliography{biblio}
\nocite{*} % Commande à supprimer

% ===================================================================================== %

\end{document}
